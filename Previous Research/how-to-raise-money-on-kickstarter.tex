\documentclass[]{article}
\usepackage{lmodern}
\usepackage{amssymb,amsmath}
\usepackage{ifxetex,ifluatex}
\usepackage{fixltx2e} % provides \textsubscript
\ifnum 0\ifxetex 1\fi\ifluatex 1\fi=0 % if pdftex
  \usepackage[T1]{fontenc}
  \usepackage[utf8]{inputenc}
\else % if luatex or xelatex
  \ifxetex
    \usepackage{mathspec}
  \else
    \usepackage{fontspec}
  \fi
  \defaultfontfeatures{Ligatures=TeX,Scale=MatchLowercase}
\fi
% use upquote if available, for straight quotes in verbatim environments
\IfFileExists{upquote.sty}{\usepackage{upquote}}{}
% use microtype if available
\IfFileExists{microtype.sty}{%
\usepackage{microtype}
\UseMicrotypeSet[protrusion]{basicmath} % disable protrusion for tt fonts
}{}
\usepackage[margin=1in]{geometry}
\usepackage{hyperref}
\hypersetup{unicode=true,
            pdftitle={How to be successful on Kickstarter},
            pdfauthor={Ann Rajaram},
            pdfborder={0 0 0},
            breaklinks=true}
\urlstyle{same}  % don't use monospace font for urls
\usepackage{color}
\usepackage{fancyvrb}
\newcommand{\VerbBar}{|}
\newcommand{\VERB}{\Verb[commandchars=\\\{\}]}
\DefineVerbatimEnvironment{Highlighting}{Verbatim}{commandchars=\\\{\}}
% Add ',fontsize=\small' for more characters per line
\usepackage{framed}
\definecolor{shadecolor}{RGB}{248,248,248}
\newenvironment{Shaded}{\begin{snugshade}}{\end{snugshade}}
\newcommand{\KeywordTok}[1]{\textcolor[rgb]{0.13,0.29,0.53}{\textbf{#1}}}
\newcommand{\DataTypeTok}[1]{\textcolor[rgb]{0.13,0.29,0.53}{#1}}
\newcommand{\DecValTok}[1]{\textcolor[rgb]{0.00,0.00,0.81}{#1}}
\newcommand{\BaseNTok}[1]{\textcolor[rgb]{0.00,0.00,0.81}{#1}}
\newcommand{\FloatTok}[1]{\textcolor[rgb]{0.00,0.00,0.81}{#1}}
\newcommand{\ConstantTok}[1]{\textcolor[rgb]{0.00,0.00,0.00}{#1}}
\newcommand{\CharTok}[1]{\textcolor[rgb]{0.31,0.60,0.02}{#1}}
\newcommand{\SpecialCharTok}[1]{\textcolor[rgb]{0.00,0.00,0.00}{#1}}
\newcommand{\StringTok}[1]{\textcolor[rgb]{0.31,0.60,0.02}{#1}}
\newcommand{\VerbatimStringTok}[1]{\textcolor[rgb]{0.31,0.60,0.02}{#1}}
\newcommand{\SpecialStringTok}[1]{\textcolor[rgb]{0.31,0.60,0.02}{#1}}
\newcommand{\ImportTok}[1]{#1}
\newcommand{\CommentTok}[1]{\textcolor[rgb]{0.56,0.35,0.01}{\textit{#1}}}
\newcommand{\DocumentationTok}[1]{\textcolor[rgb]{0.56,0.35,0.01}{\textbf{\textit{#1}}}}
\newcommand{\AnnotationTok}[1]{\textcolor[rgb]{0.56,0.35,0.01}{\textbf{\textit{#1}}}}
\newcommand{\CommentVarTok}[1]{\textcolor[rgb]{0.56,0.35,0.01}{\textbf{\textit{#1}}}}
\newcommand{\OtherTok}[1]{\textcolor[rgb]{0.56,0.35,0.01}{#1}}
\newcommand{\FunctionTok}[1]{\textcolor[rgb]{0.00,0.00,0.00}{#1}}
\newcommand{\VariableTok}[1]{\textcolor[rgb]{0.00,0.00,0.00}{#1}}
\newcommand{\ControlFlowTok}[1]{\textcolor[rgb]{0.13,0.29,0.53}{\textbf{#1}}}
\newcommand{\OperatorTok}[1]{\textcolor[rgb]{0.81,0.36,0.00}{\textbf{#1}}}
\newcommand{\BuiltInTok}[1]{#1}
\newcommand{\ExtensionTok}[1]{#1}
\newcommand{\PreprocessorTok}[1]{\textcolor[rgb]{0.56,0.35,0.01}{\textit{#1}}}
\newcommand{\AttributeTok}[1]{\textcolor[rgb]{0.77,0.63,0.00}{#1}}
\newcommand{\RegionMarkerTok}[1]{#1}
\newcommand{\InformationTok}[1]{\textcolor[rgb]{0.56,0.35,0.01}{\textbf{\textit{#1}}}}
\newcommand{\WarningTok}[1]{\textcolor[rgb]{0.56,0.35,0.01}{\textbf{\textit{#1}}}}
\newcommand{\AlertTok}[1]{\textcolor[rgb]{0.94,0.16,0.16}{#1}}
\newcommand{\ErrorTok}[1]{\textcolor[rgb]{0.64,0.00,0.00}{\textbf{#1}}}
\newcommand{\NormalTok}[1]{#1}
\usepackage{graphicx,grffile}
\makeatletter
\def\maxwidth{\ifdim\Gin@nat@width>\linewidth\linewidth\else\Gin@nat@width\fi}
\def\maxheight{\ifdim\Gin@nat@height>\textheight\textheight\else\Gin@nat@height\fi}
\makeatother
% Scale images if necessary, so that they will not overflow the page
% margins by default, and it is still possible to overwrite the defaults
% using explicit options in \includegraphics[width, height, ...]{}
\setkeys{Gin}{width=\maxwidth,height=\maxheight,keepaspectratio}
\IfFileExists{parskip.sty}{%
\usepackage{parskip}
}{% else
\setlength{\parindent}{0pt}
\setlength{\parskip}{6pt plus 2pt minus 1pt}
}
\setlength{\emergencystretch}{3em}  % prevent overfull lines
\providecommand{\tightlist}{%
  \setlength{\itemsep}{0pt}\setlength{\parskip}{0pt}}
\setcounter{secnumdepth}{0}
% Redefines (sub)paragraphs to behave more like sections
\ifx\paragraph\undefined\else
\let\oldparagraph\paragraph
\renewcommand{\paragraph}[1]{\oldparagraph{#1}\mbox{}}
\fi
\ifx\subparagraph\undefined\else
\let\oldsubparagraph\subparagraph
\renewcommand{\subparagraph}[1]{\oldsubparagraph{#1}\mbox{}}
\fi

%%% Use protect on footnotes to avoid problems with footnotes in titles
\let\rmarkdownfootnote\footnote%
\def\footnote{\protect\rmarkdownfootnote}

%%% Change title format to be more compact
\usepackage{titling}

% Create subtitle command for use in maketitle
\newcommand{\subtitle}[1]{
  \posttitle{
    \begin{center}\large#1\end{center}
    }
}

\setlength{\droptitle}{-2em}

  \title{How to be successful on Kickstarter}
    \pretitle{\vspace{\droptitle}\centering\huge}
  \posttitle{\par}
    \author{Ann Rajaram}
    \preauthor{\centering\large\emph}
  \postauthor{\par}
      \predate{\centering\large\emph}
  \postdate{\par}
    \date{January 29, 2018}


\begin{document}
\maketitle

\subsection{}\label{section}

In this tutorial, we will explore the characterisitcs of projects on
Kickstarter and try to understand what separates the winners from the
projects that failed to reach their funding goals.

\subsection{\texorpdfstring{{Qs for Exploratory
Analysis:}}{Qs for Exploratory Analysis:}}\label{qs-for-exploratory-analysis}

We will start our analysis with the aim of answering the following
questions:

\begin{enumerate}
\def\labelenumi{\arabic{enumi}.}
\tightlist
\item
  How many projects were successful on Kickstarter, by year and
  category.
\item
  Which sub-categories raised the most amount of money?
\item
  Projects originate from which countries?
\item
  How many projects exceeded their funding goal by 50\% or more?
\item
  Did any projects reach \$100,000 or more? \$1,000,000 or higher?
\item
  What was the average amount contributed by each backer, and how does
  this change over time? Does this amount differ with categories?
\item
  What is the average funding period?
\end{enumerate}

\subsection{\texorpdfstring{{Predicting success
rates:}}{Predicting success rates:}}\label{predicting-success-rates}

Using the answers from the above questions, we will try to create a
model that can predict which projects are most likely to be successful.
If you like this tutorial, feel free to fork the script. And dont forget
to upvote the kernel! :)

\subsection{\texorpdfstring{{Step1 - Data
Pre-processing}}{Step1 - Data Pre-processing}}\label{step1---data-pre-processing}

\subsubsection{a) Let us take a look at the input dataset
:}\label{a-let-us-take-a-look-at-the-input-dataset}

\begin{verbatim}
## 'data.frame':    378661 obs. of  15 variables:
##  $ ID              : int  1000002330 1000003930 1000004038 1000007540 1000011046 1000014025 1000023410 1000030581 1000034518 100004195 ...
##  $ name            : chr  "The Songs of Adelaide & Abullah" "Greeting From Earth: ZGAC Arts Capsule For ET" "Where is Hank?" "ToshiCapital Rekordz Needs Help to Complete Album" ...
##  $ category        : chr  "Poetry" "Narrative Film" "Narrative Film" "Music" ...
##  $ main_category   : chr  "Publishing" "Film & Video" "Film & Video" "Music" ...
##  $ currency        : chr  "GBP" "USD" "USD" "USD" ...
##  $ deadline        : chr  "2015-10-09" "2017-11-01" "2013-02-26" "2012-04-16" ...
##  $ goal            : num  1000 30000 45000 5000 19500 50000 1000 25000 125000 65000 ...
##  $ launched        : chr  "2015-08-11 12:12:28" "2017-09-02 04:43:57" "2013-01-12 00:20:50" "2012-03-17 03:24:11" ...
##  $ pledged         : num  0 2421 220 1 1283 ...
##  $ state           : chr  "failed" "failed" "failed" "failed" ...
##  $ backers         : int  0 15 3 1 14 224 16 40 58 43 ...
##  $ country         : chr  "GB" "US" "US" "US" ...
##  $ usd.pledged     : num  0 100 220 1 1283 ...
##  $ usd_pledged_real: num  0 2421 220 1 1283 ...
##  $ usd_goal_real   : num  1534 30000 45000 5000 19500 ...
\end{verbatim}

The projects are divided into main and sub-categories. The pledged
amount ``usd\_pledged'' has an equivalent value converted to USD, called
``usd\_pledged\_real''. However, the goal amount does not have this
conversion. So for now, we will use the amounts as is.

We can see how many people are backing each individual project using the
column, ``backers''.

\subsubsection{b) Now let us look at the first 5
records:}\label{b-now-let-us-look-at-the-first-5-records}

The name doesn't really indicate any specific pattern although it might
be interesting to see if longer names have better success rates. Not
pursuing that angle at this time, though.

\begin{verbatim}
##           ID                                                       name
## 1 1000002330                            The Songs of Adelaide & Abullah
## 2 1000003930              Greeting From Earth: ZGAC Arts Capsule For ET
## 3 1000004038                                             Where is Hank?
## 4 1000007540          ToshiCapital Rekordz Needs Help to Complete Album
## 5 1000011046 Community Film Project: The Art of Neighborhood Filmmaking
## 6 1000014025                                       Monarch Espresso Bar
##         category main_category currency   deadline  goal
## 1         Poetry    Publishing      GBP 2015-10-09  1000
## 2 Narrative Film  Film & Video      USD 2017-11-01 30000
## 3 Narrative Film  Film & Video      USD 2013-02-26 45000
## 4          Music         Music      USD 2012-04-16  5000
## 5   Film & Video  Film & Video      USD 2015-08-29 19500
## 6    Restaurants          Food      USD 2016-04-01 50000
##              launched pledged      state backers country usd.pledged
## 1 2015-08-11 12:12:28       0     failed       0      GB           0
## 2 2017-09-02 04:43:57    2421     failed      15      US         100
## 3 2013-01-12 00:20:50     220     failed       3      US         220
## 4 2012-03-17 03:24:11       1     failed       1      US           1
## 5 2015-07-04 08:35:03    1283   canceled      14      US        1283
## 6 2016-02-26 13:38:27   52375 successful     224      US       52375
##   usd_pledged_real usd_goal_real
## 1                0       1533.95
## 2             2421      30000.00
## 3              220      45000.00
## 4                1       5000.00
## 5             1283      19500.00
## 6            52375      50000.00
\end{verbatim}

\subsubsection{c) Looking for missing
values:}\label{c-looking-for-missing-values}

Hurrah, a really clean dataset, even after searching for ``empty''
strings. :)

\begin{Shaded}
\begin{Highlighting}[]
\CommentTok{# Check for NAs:}
\KeywordTok{sapply}\NormalTok{(ksdf, }\ControlFlowTok{function}\NormalTok{(x) }\KeywordTok{sum}\NormalTok{(}\KeywordTok{is.na}\NormalTok{(x)))}
\end{Highlighting}
\end{Shaded}

\begin{verbatim}
##               ID             name         category    main_category 
##                0                0                0                0 
##         currency         deadline             goal         launched 
##                0                0                0                0 
##          pledged            state          backers          country 
##                0                0                0                0 
##      usd.pledged usd_pledged_real    usd_goal_real 
##             3797                0                0
\end{verbatim}

\begin{Shaded}
\begin{Highlighting}[]
\CommentTok{# Check for empty strings:}
\KeywordTok{nrow}\NormalTok{(}\KeywordTok{subset}\NormalTok{(ksdf, }\KeywordTok{is.na}\NormalTok{(ksdf}\OperatorTok{$}\NormalTok{name)))}
\end{Highlighting}
\end{Shaded}

\begin{verbatim}
## [1] 0
\end{verbatim}

\begin{Shaded}
\begin{Highlighting}[]
\KeywordTok{nrow}\NormalTok{(}\KeywordTok{subset}\NormalTok{(ksdf, }\KeywordTok{is.na}\NormalTok{(ksdf}\OperatorTok{$}\NormalTok{category)))}
\end{Highlighting}
\end{Shaded}

\begin{verbatim}
## [1] 0
\end{verbatim}

\begin{Shaded}
\begin{Highlighting}[]
\KeywordTok{nrow}\NormalTok{(}\KeywordTok{subset}\NormalTok{(ksdf, }\KeywordTok{is.na}\NormalTok{(ksdf}\OperatorTok{$}\NormalTok{main_category)))}
\end{Highlighting}
\end{Shaded}

\begin{verbatim}
## [1] 0
\end{verbatim}

\begin{Shaded}
\begin{Highlighting}[]
\KeywordTok{nrow}\NormalTok{(}\KeywordTok{subset}\NormalTok{(ksdf, }\KeywordTok{is.na}\NormalTok{(ksdf}\OperatorTok{$}\NormalTok{state)))}
\end{Highlighting}
\end{Shaded}

\begin{verbatim}
## [1] 0
\end{verbatim}

\subsubsection{\texorpdfstring{d) Date Formatting and
splitting:}{d) Date Formatting and splitting: }}\label{d-date-formatting-and-splitting}

We have two dates in our dataset - ``launch date'' and ``deadline
date''.We convert them from strings to date format. We also split these
dates into the respective year and month columns, so that we can plot
variations over time. So we will now have 4 new columns: launch\_year,
launch\_month, deadline\_year and deadline\_month.

\begin{verbatim}
##           ID                                                       name
## 1 1000002330                            The Songs of Adelaide & Abullah
## 2 1000003930              Greeting From Earth: ZGAC Arts Capsule For ET
## 3 1000004038                                             Where is Hank?
## 4 1000007540          ToshiCapital Rekordz Needs Help to Complete Album
## 5 1000011046 Community Film Project: The Art of Neighborhood Filmmaking
##   main_category launch_year launch_mth final_year final_mth
## 1    Publishing        2015    2015-08       2015   2015-10
## 2  Film & Video        2017    2017-09       2017   2017-11
## 3  Film & Video        2013    2013-01       2013   2013-02
## 4         Music        2012    2012-03       2012   2012-04
## 5  Film & Video        2015    2015-07       2015   2015-08
\end{verbatim}

\subsection{\texorpdfstring{{Exploratory
analysis:}}{Exploratory analysis:}}\label{exploratory-analysis}

\subsubsection{a) How many projects are
successful?}\label{a-how-many-projects-are-successful}

\begin{Shaded}
\begin{Highlighting}[]
\KeywordTok{prop.table}\NormalTok{(}\KeywordTok{table}\NormalTok{(ksdf}\OperatorTok{$}\NormalTok{state))}\OperatorTok{*}\DecValTok{100}
\end{Highlighting}
\end{Shaded}

\begin{verbatim}
## 
##   canceled     failed       live successful  suspended  undefined 
## 10.2410864 52.2153060  0.7391836 35.3762336  0.4875073  0.9406831
\end{verbatim}

We see that ``failed'' and ``successful'' are the two main categories,
comprising \textasciitilde{}88\% of our dataset. Sadly we do not know
why some projects are marked ``undefined'' or ``canceled''. ``live''"
projects are those where the deadlines have not yet passed, although a
few among them are already achieved their goal. Surprisingly, some
`canceled' projects had also met their goals (pledged\_amount
\textgreater{}= goal). Since these other categories are a very small
portion of the dataset, we will subset and only consider records with
satus ``failed'' or ``successful'' for the rest of the analysis.

\subsubsection{b) How many countries have projects on
kickstarter?}\label{b-how-many-countries-have-projects-on-kickstarter}

\begin{verbatim}
## 
##     AT     AU     BE     CA     CH     DE     DK     ES     FR     GB 
##    485   6616    523  12370    652   3436    926   1873   2520  29454 
##     HK     IE     IT     JP     LU     MX  N,0""     NL     NO     NZ 
##    477    683   2369     23     57   1411    210   2411    582   1274 
##     SE     SG     US 
##   1509    454 261360
\end{verbatim}

We see projects are overwhelmingly US. Some country names have the tag
N,0``'', so marking them as unknown.

\subsubsection{c) Number of projects launched per
year:}\label{c-number-of-projects-launched-per-year}

\begin{verbatim}
## 
##  2009  2010  2011  2012  2013  2014  2015  2016  2017 
##  1179  9577 24049 38480 41101 59306 65272 49292 43419
\end{verbatim}

Looks like some records say dates like 1970, which does not look right.
So we discard any records with a launch / deadline year before 2009.
Plotting the counts per year on a graphs: \textless{} br
/\textgreater{}From the graph below, it looks like the count of projects
peaked in 2015, then went down. However, this should NOT be taken as an
indicator of success rates.
\includegraphics{how-to-raise-money-on-kickstarter_files/figure-latex/unnamed-chunk-11-1.pdf}

Drilling down a bit more to see count of projects by main\_category.
\includegraphics{how-to-raise-money-on-kickstarter_files/figure-latex/unnamed-chunk-12-1.pdf}
Over the years, maximum number of projects have been in the categories:

\begin{enumerate}
\def\labelenumi{\arabic{enumi}.}
\tightlist
\item
  Film \& Video
\item
  Music
\item
  Publishing
\end{enumerate}

\subsubsection{d) Number of projects by sub-category: (Top 20
only)}\label{d-number-of-projects-by-sub-category-top-20-only}

\includegraphics{how-to-raise-money-on-kickstarter_files/figure-latex/unnamed-chunk-13-1.pdf}
The Top 5 sub-categories are:

\begin{enumerate}
\def\labelenumi{\arabic{enumi}.}
\tightlist
\item
  Product Design
\item
  Documentary
\item
  Music
\item
  Tabletop Games (interesting!!!)
\item
  Shorts (really?! )
\end{enumerate}

 Let us now see ``Status'' of projects for these Top 5 sub\_categories:
From the graph below, we see that for category ``shorts'' and ``tabletop
games'' there are more successfull projects than failed ones.
\includegraphics{how-to-raise-money-on-kickstarter_files/figure-latex/unnamed-chunk-14-1.pdf}

\subsubsection{e) Backers by category and
sub-category:}\label{e-backers-by-category-and-sub-category}

\includegraphics{how-to-raise-money-on-kickstarter_files/figure-latex/unnamed-chunk-15-1.pdf}

 Since there are a lot of sub-categories, let us explore the
sub-categories under the main theme ``Design''
\includegraphics{how-to-raise-money-on-kickstarter_files/figure-latex/unnamed-chunk-16-1.pdf}

Product design is not just the sub-category with the highest count of
projects, but also the category with the highest success ratio.

\subsubsection{f) add flag to see how many got funded more than the
goal.}\label{f-add-flag-to-see-how-many-got-funded-more-than-the-goal.}

\begin{Shaded}
\begin{Highlighting}[]
\NormalTok{ksdf}\OperatorTok{$}\NormalTok{goal_flag <-}\StringTok{ }\KeywordTok{ifelse}\NormalTok{(ksdf}\OperatorTok{$}\NormalTok{pledged }\OperatorTok{>=}\StringTok{ }\NormalTok{ksdf}\OperatorTok{$}\NormalTok{goal, }\DecValTok{1}\NormalTok{, }\DecValTok{0}\NormalTok{)}
\KeywordTok{prop.table}\NormalTok{(}\KeywordTok{table}\NormalTok{(ksdf}\OperatorTok{$}\NormalTok{goal_flag))}\OperatorTok{*}\DecValTok{100}
\end{Highlighting}
\end{Shaded}

\begin{verbatim}
## 
##        0        1 
## 59.61197 40.38803
\end{verbatim}

So \textasciitilde{}40\% of projects reached or surpassed their goal,
which matches the number of successful projects .

\subsubsection{g) Calculate average contribution per
backer:}\label{g-calculate-average-contribution-per-backer}

From the mean, median and max values we quickly see that the median
amount contributed by each backer is only \textasciitilde{}\$40 whereas
the mean is higher due to the extreme positive values. The max amount by
a single backer is \textasciitilde{}\$5000.

\begin{Shaded}
\begin{Highlighting}[]
\KeywordTok{summary}\NormalTok{(ksdf}\OperatorTok{$}\NormalTok{contrib)}
\end{Highlighting}
\end{Shaded}

\begin{verbatim}
##     Min.  1st Qu.   Median     Mean  3rd Qu.     Max. 
##     0.00    16.00    41.78    73.35    78.00 50000.00
\end{verbatim}

\begin{Shaded}
\begin{Highlighting}[]
\KeywordTok{hist}\NormalTok{(n}\OperatorTok{$}\NormalTok{contrib, }\DataTypeTok{main =} \StringTok{"Histogram for number of contributors"}\NormalTok{)}
\end{Highlighting}
\end{Shaded}

\includegraphics{how-to-raise-money-on-kickstarter_files/figure-latex/unnamed-chunk-19-1.pdf}

\subsubsection{h) Calculate reach\_ratio}\label{h-calculate-reach_ratio}

 The amount per backer is a good start, but what if the goal amount
itself is only \$1000? Then an average contribution per backer of \$50
impies we only need 20 backers. So to better understand the probability
of a project's success, we create a derived metric called
``reach\_ratio''. This takes the average user contribution and compares
it against the goal fund amount.

\begin{verbatim}
##      Min.   1st Qu.    Median      Mean   3rd Qu.      Max. 
##      0.00      0.16      0.75      6.17      2.16 166666.67
\end{verbatim}

 We see the median reach\_ratio is \textless{}1\%. Only in the third
quartile do we even touch 2\%! Clearly most projects have a very low
reach ratio. We could subset for ``successful'' projects only and check
if the reach\_ratio is higher.
\includegraphics{how-to-raise-money-on-kickstarter_files/figure-latex/unnamed-chunk-21-1.pdf}

\subsubsection{i) Number of days to achieve
goal:}\label{i-number-of-days-to-achieve-goal}

\includegraphics{how-to-raise-money-on-kickstarter_files/figure-latex/unnamed-chunk-22-1.pdf}

\subsection{\texorpdfstring{{Predictive
Analystics:}}{Predictive Analystics:}}\label{predictive-analystics}

We will apply a very simple decision tree algorithm to our dataset.
Since we do not have a separate ``test'' set, we will split the input
dataframe into 2 parts (70/30 split). We will use the smaller set to
test the accuracy of out algorithm.

\begin{Shaded}
\begin{Highlighting}[]
\NormalTok{ksdf}\OperatorTok{$}\NormalTok{status =}\StringTok{ }\KeywordTok{ifelse}\NormalTok{(ksdf}\OperatorTok{$}\NormalTok{state }\OperatorTok{==}\StringTok{ 'failed'}\NormalTok{, }\DecValTok{0}\NormalTok{, }\DecValTok{1}\NormalTok{)}

\NormalTok{## 70% of the sample size}
\NormalTok{smp_size <-}\StringTok{ }\KeywordTok{floor}\NormalTok{(}\FloatTok{0.7} \OperatorTok{*}\StringTok{ }\KeywordTok{nrow}\NormalTok{(ksdf))}

\NormalTok{## set the seed to make your partition reproductible}
\KeywordTok{set.seed}\NormalTok{(}\DecValTok{486}\NormalTok{)}
\NormalTok{train_ind <-}\StringTok{ }\KeywordTok{sample}\NormalTok{(}\KeywordTok{seq_len}\NormalTok{(}\KeywordTok{nrow}\NormalTok{(ksdf)), }\DataTypeTok{size =}\NormalTok{ smp_size)}

\NormalTok{train <-}\StringTok{ }\NormalTok{ksdf[train_ind, ]}
\NormalTok{test <-}\StringTok{ }\NormalTok{ksdf[}\OperatorTok{-}\NormalTok{train_ind, ]}
\end{Highlighting}
\end{Shaded}

\begin{Shaded}
\begin{Highlighting}[]
\KeywordTok{library}\NormalTok{(tree)}
\NormalTok{tree1 <-}\StringTok{ }\KeywordTok{tree}\NormalTok{(status }\OperatorTok{~}\StringTok{ }\NormalTok{goal }\OperatorTok{+}\StringTok{ }\NormalTok{reach_ratio }\OperatorTok{+}\StringTok{ }\NormalTok{category }\OperatorTok{+}\StringTok{ }\NormalTok{backers }\OperatorTok{+}\StringTok{ }\NormalTok{country }\OperatorTok{+}\StringTok{ }\NormalTok{launch_year , }\DataTypeTok{data =}\NormalTok{ train)}
\end{Highlighting}
\end{Shaded}

\begin{verbatim}
## Warning in tree(status ~ goal + reach_ratio + category + backers + country
## + : NAs introduced by coercion
\end{verbatim}

\begin{Shaded}
\begin{Highlighting}[]
\KeywordTok{summary}\NormalTok{(tree1)}
\end{Highlighting}
\end{Shaded}

\begin{verbatim}
## 
## Regression tree:
## tree(formula = status ~ goal + reach_ratio + category + backers + 
##     country + launch_year, data = train)
## Variables actually used in tree construction:
## [1] "backers"     "reach_ratio"
## Number of terminal nodes:  9 
## Residual mean deviance:  0.02429 = 5640 / 232200 
## Distribution of residuals:
##       Min.    1st Qu.     Median       Mean    3rd Qu.       Max. 
## -0.9816000 -0.0006945 -0.0006945  0.0000000  0.0410400  0.9993000
\end{verbatim}

Taking a peek at the decision tree rules:

\begin{Shaded}
\begin{Highlighting}[]
\KeywordTok{plot}\NormalTok{(tree1)}
\KeywordTok{text}\NormalTok{(tree1 ,}\DataTypeTok{pretty =}\DecValTok{0}\NormalTok{)}
\end{Highlighting}
\end{Shaded}

\includegraphics{how-to-raise-money-on-kickstarter_files/figure-latex/unnamed-chunk-25-1.pdf}

\begin{Shaded}
\begin{Highlighting}[]
\NormalTok{tree1}
\end{Highlighting}
\end{Shaded}

\begin{verbatim}
## node), split, n, deviance, yval
##       * denotes terminal node
## 
##  1) root 232172 55900.00 0.4039000  
##    2) backers < 17.5 121911  9326.00 0.0834600  
##      4) reach_ratio < 5.88118 107986    74.95 0.0006945 *
##      5) reach_ratio > 5.88118 13925  2774.00 0.7253000  
##       10) backers < 5.5 4723  1031.00 0.3218000  
##         20) reach_ratio < 19.9667 2792     0.00 0.0000000 *
##         21) reach_ratio > 19.9667 1931   323.50 0.7872000 *
##       11) backers > 5.5 9202   580.00 0.9324000 *
##    3) backers > 17.5 110261 20210.00 0.7583000  
##      6) reach_ratio < 0.79672 40852 10190.00 0.5217000  
##       12) backers < 128.5 16858    39.91 0.0023730 *
##       13) backers > 128.5 23994  2413.00 0.8866000 *
##      7) reach_ratio > 0.79672 69409  6383.00 0.8975000  
##       14) backers < 35.5 20535  3836.00 0.7514000  
##         28) reach_ratio < 2.85458 4816     0.00 0.0000000 *
##         29) reach_ratio > 2.85458 15719   284.60 0.9816000 *
##       15) backers > 35.5 48874  1924.00 0.9590000 *
\end{verbatim}

Re-applying the tree rules to the training set itself, we can validate
our model:

\begin{Shaded}
\begin{Highlighting}[]
\NormalTok{Predt <-}\StringTok{ }\KeywordTok{predict}\NormalTok{(tree1, train)}
\end{Highlighting}
\end{Shaded}

\begin{verbatim}
## Warning in pred1.tree(object, tree.matrix(newdata)): NAs introduced by
## coercion
\end{verbatim}

\begin{Shaded}
\begin{Highlighting}[]
\NormalTok{validf <-}\StringTok{ }\KeywordTok{data.frame}\NormalTok{( }\DataTypeTok{kickstarter_id =}\NormalTok{ train}\OperatorTok{$}\NormalTok{ID, }\DataTypeTok{orig_status =}\NormalTok{ train}\OperatorTok{$}\NormalTok{status, }\DataTypeTok{new_status =}\NormalTok{ Predt)}
\NormalTok{validf}\OperatorTok{$}\NormalTok{new =}\StringTok{ }\KeywordTok{ifelse}\NormalTok{(validf}\OperatorTok{$}\NormalTok{new_status }\OperatorTok{<}\StringTok{ }\FloatTok{0.5}\NormalTok{, }\DecValTok{0}\NormalTok{, }\DecValTok{1}\NormalTok{)}

\CommentTok{# contingency Tables:}
\KeywordTok{table}\NormalTok{(validf}\OperatorTok{$}\NormalTok{orig_status, validf}\OperatorTok{$}\NormalTok{new)}
\end{Highlighting}
\end{Shaded}

\begin{verbatim}
##    
##          0      1
##   0 132337   6051
##   1    115  93669
\end{verbatim}

\begin{Shaded}
\begin{Highlighting}[]
\CommentTok{# Area under the curve}
\KeywordTok{library}\NormalTok{(pROC)}
\KeywordTok{auc}\NormalTok{(validf}\OperatorTok{$}\NormalTok{orig_status, validf}\OperatorTok{$}\NormalTok{new)}
\end{Highlighting}
\end{Shaded}

\begin{verbatim}
## Area under the curve: 0.9775
\end{verbatim}

From the above tables, we see that the error rate = \textasciitilde{}3\%
and area under curve \textgreater{}= 97\%

Finally applying the tree rules to the test set, we get the following
stats:

\begin{Shaded}
\begin{Highlighting}[]
\NormalTok{Pred1 <-}\StringTok{ }\KeywordTok{predict}\NormalTok{(tree1, test)}
\end{Highlighting}
\end{Shaded}

\begin{verbatim}
## Warning in pred1.tree(object, tree.matrix(newdata)): NAs introduced by
## coercion
\end{verbatim}

\begin{Shaded}
\begin{Highlighting}[]
\NormalTok{chkdf <-}\StringTok{ }\KeywordTok{data.frame}\NormalTok{( }\DataTypeTok{kickstarter_id =}\NormalTok{ test}\OperatorTok{$}\NormalTok{ID, }\DataTypeTok{orig_status =}\NormalTok{ test}\OperatorTok{$}\NormalTok{status, }\DataTypeTok{new_status =}\NormalTok{ Pred1)}
\NormalTok{chkdf}\OperatorTok{$}\NormalTok{new =}\StringTok{ }\KeywordTok{ifelse}\NormalTok{(chkdf}\OperatorTok{$}\NormalTok{new_status }\OperatorTok{<}\StringTok{ }\FloatTok{0.5}\NormalTok{, }\DecValTok{0}\NormalTok{, }\DecValTok{1}\NormalTok{)}

\CommentTok{# contingency Tables:}
\KeywordTok{table}\NormalTok{(chkdf}\OperatorTok{$}\NormalTok{orig_status, chkdf}\OperatorTok{$}\NormalTok{new)}
\end{Highlighting}
\end{Shaded}

\begin{verbatim}
##    
##         0     1
##   0 56771  2560
##   1    42 40130
\end{verbatim}

\begin{Shaded}
\begin{Highlighting}[]
\CommentTok{# Area under the curve}
\KeywordTok{library}\NormalTok{(pROC)}
\KeywordTok{auc}\NormalTok{(chkdf}\OperatorTok{$}\NormalTok{orig_status, chkdf}\OperatorTok{$}\NormalTok{new)}
\end{Highlighting}
\end{Shaded}

\begin{verbatim}
## Area under the curve: 0.9779
\end{verbatim}

From the above tables, we see that still the error rate =
\textasciitilde{}3\% and area under curve \textgreater{}= 97\%

\subsection{\texorpdfstring{{Conclustion:}}{Conclustion:}}\label{conclustion}

Thus in this tutorial, we explored the factors that contribtue to a
project's success. Main theme and sub-category were important, but the
number of backers and ``reach\_ratio'' were found to be most critical.
If a founder wanted to gauge their probability of success, they could
measure their ``reach-ratio'' halfway to the deadline, or perhaps when
25\% of the timeline is complete. If the numbers are lower, it means
they need to double down and use promotions/social media marketing to
get more backers and funding.

If you liked this tutorial, feel free to fork the script. And dont
forget to upvote the kernel! :)


\end{document}
